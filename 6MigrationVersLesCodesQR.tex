\newpage

\section{Migration vers les codes QR}

Il existe plusieurs avantages et désavantages pour l'utilisation des codes-QR, voici une liste non exhaustive des points les plus importants que nous avons relevés.

\subsection{Avantages}

\begin{itemize}
\item Il existe déjà de nombreuses applications capables de lire les QR Codes sur de nombreuses plateformes
\end{itemize}
Cela évite de devoir développer une application par nous même pour chaque plateforme.

\begin{itemize}
\item Permet d'effectuer un virement direct via son téléphone portable
\end{itemize}
Cela permetrait éventuellement de payer directement les médicaments dans les pharmacies ou les hopitaux grâce à un simple scan du QR code.

\begin{itemize}
\item Les codes QR peuvent généralement être lus même si 25$\%$ de la surface du QR code est effacée
\end{itemize}
Ce pourcentage de surface pouvant être abimé est excellent. Cela permet d'assurer une lecture des codes QR avec une grande efficacité.


\subsection{Désavantages}

\begin{itemize}
\item La quantité d'information que l'on peut stockée est limitée à environ 500 mots
\end{itemize}
Cette quantité est moins importante que lors du stockage d'informations dans un CB2D. Hors le stockage d'information est le but premier de l'utilisation des CB2D.