\section*{Conclusion}

Le fonctionnement du programme est objectivement satisfaisant, le programme est tout à fait conforme aux spécifications du cahier des charges établi au début du projet. En effet, celui-ci est capable de lire un CB2D chargé à partir d'un fichier image spécifié par l'utilisateur mais également de générer un CB2D à partir de texte fourni par l'utilisateur, le tout en moins de deux secondes. De plus, le programme intègre un algorithme de correction des CB2D qui va permettre, avant la lecture, de vérifier que celui-ci n'a pas été endommagé depuis son impression.

De plus, le programme est assez facile d'utilisation et très intuitif. L'utilisateur est guidé tout au long de la procédure de lecture ou de génération des CB2D ce qui permet à n'importe qui d'utiliser le programme à pleine efficacité sans aucun problème.

En guise de conclusion, les difficultés rencontrées lors du projet sont l'assimilation des interfaces et des classes dans le code afin d'en extraire l'essentiel.