\newpage

\section{Fonctionnalité du code}

Le programme est séparé en plusieurs blocs distincts. Ce rapport ne détaillera cependant que les trois plus important d'entres eux. En effet, ces trois blocs sont le bloc dédié à la getsion de l'interaction entre l'utilisateur et le programme et ceux dédiés à la génération et la lecture des code-barres CB2D.

\subsection{CodeBarManager}

Cette partie du programme a pour but, comme indiqué ci dessus, de gérer les interactions entre l'utilisateur et le programme. L'utilisateur a en effet plusieurs choix à faire durant l'utilisation du programme : 

\begin{itemize}
\item Lecture d'un code-barre CB2D
	\begin{enumerate}
	\item Entrer le chemin d'accès au code-barre CB2D
	\end{enumerate}
\item Génération d'un code-barre CB2D
	\begin{enumerate}
	\item Entrer le texte que l'on veut stocker dans le CB2D
	\item Entrer le type de format que l'on veut utiliser
	\item Entrer le type de compression que l'on veut appliquer
	\item Entrer le chemin d'accès où le fichier sera sauvegardé
	\end{enumerate}
\end{itemize}

Chaque fois que l'utilisateur doit entrer quelque chose à la console, nous vérifions que ce qu'il entre est correcte et n'amènera pas d'erreur dans le programme. 
\\
Mais c'est également dans cette partie du programme que : 
\begin{itemize}
\item Les informations vont être extraites (non décodées !) du CB2D à partir du fichier image
\item Le CB2D va être dessiner et sauvegarder dans un fichier au format .png
\end{itemize}

\subsection{Encodeur}

Cette partie du programme a pour fonction d'encoder le message fournit par l'utilisateur. C'est grâce à ce bloc du programme que le code va être transformé d'un simple texte en langage binaire prêt à être dessiné en CB2D. Pour cela, le programme va non seulement transformer le texte en binaire et le placer dans une matrice, mais il va également mettre en place l'algorithme de correction du futur CB2D. 
\subsection{Decodeur}

Cette dernière partie a, quant à elle, pour but de décoder les informations tirées d'un CB2D. 
En effet, lors de la lecture d'un CB2D, le programme va recevoir des informations sous forme binaire et il devra donc décoder ces informations afin d'en tirer les données utiles à l'utilisateur. Pour cela, le programme va d'abord vérifier si il n'y a pas d'erreurs dans la matrice de bits. Ensuite, si il n'y a pas de fautes ou si il a pu les corriger, le programme va traduire la chaine de bits obtenue en texte.