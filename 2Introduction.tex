
\section{Introduction}

Dans le cadre du cours de résolution informatique de problèmes, la société N-Side a demandé aux étudiant de développer un programme capable de lire et de générer des code-barres au format spécifique CB2D. Le rapport suivant à pour but d'expliquer dans le détail les spécifications du programme ainsi que la manière de l'utiliser. Tout d'abord, le rapport détaillera le cahier des charges, avec ses fonctions principales et ses contraintes. Ensuite, le rapport montrera la façon dont l'utilisateur pourra intéragir avec le programme et les différentes options qui se présenterons à lui. Par après, le rapport expliquera les différentes fonctionnalités du code, les choix qui ont été fait et les justifications de ces choix. 

